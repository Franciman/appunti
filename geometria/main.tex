\documentclass[12pt, a4paper]{article}
\usepackage[utf8]{inputenc}
\usepackage[italian]{babel}
\usepackage{fontspec}
\setmainfont{Garamond}

\usepackage[margin=2cm]{geometry}

\usepackage{amsmath}
\usepackage{amsfonts}
\usepackage{amsthm}

\theoremstyle{definition}
\newtheorem{defi}{Definizione}
\newtheorem{thm}{Teorema}
\newtheorem{lem}{Lemma}
\newtheorem{cor}{Corollario}

\usepackage{xcolor}
\definecolor{colorelink}{HTML}{ca2c92}
\usepackage{hyperref}
\hypersetup{
    colorlinks=true,
    linkcolor=colorelink
}

\title{Appunti Geometria}
\author{Francesco}

\begin{document}
\maketitle
\tableofcontents
\newpage

\section{Costruzioni di algebra multilineare}

\subsection{Moduli liberi} \label{sec:modulo-libero}
\begin{defi}
\label{def:modulo-libero}
Sia $R$ un anello commutativo unitario, $M$ un $R$-modulo e $X \subseteq M$ una sua parte.
$M$ si dice \emph{libero} su $X$ se è privo di torsione ed è somma diretta degli $R$-moduli generati dagli elementi di $X$, ovvero:
$$M = \oplus_{x \in X} \langle x \rangle_R$$
Dove $\langle x \rangle_R = \{ r \cdot x : r \in R \}$ è l'$R$ sottomodulo di $M$ generato da $x$.
\end{defi}

Gli $R$-moduli liberi sono caratterizzati dalla seguente proprietà universale,

\begin{lem}
\label{thm:prop-universale-modulo-libero}
Siano $M$ un $R$-modulo libero su $X$, $N$ un altro $R$-modulo e sia $f : X \to N$,
esiste ed è \emph{unico} un omomorfismo $R$-lineare $\bar{f} : M \to N$ che estende $f$.
\end{lem}

\begin{proof}
Iniziamo a dare una definizione per $\bar{f}$ e poi mostriamo che essa è l'unica funzione possibile con le proprietà richieste.

Siccome $M$ è privo di torsione e si ha $M = \oplus_{x \in X} \langle x \rangle_R$, ogni elemento di $M$ si scrive in \emph{un unico modo} come combinazione lineare a coefficienti in $R$ di un numero finito di elementi di $X$.

Allora la seguente definizione è ben posta: $\bar{f}(\sum\limits_{i = 0}^{n} r_i x_i) = \sum\limits_{i = 0}^{n} r_i f(x_i)$.

Ovviamente quella che abbiamo definito è una funzione lineare che coincide con $f$ su $X$, quindi ora supponiamo che esista anche un'altra funzione $g : M \to N$ con le stesse proprietà,
siccome $\bar{f}$ e $g$ coincidono su $X$, visto che entrambe estendono $f$ ed essendo entrambi omomorfismo $R$-lineari, si ha che, coincidendo su $X$, coincidono anche sull' $R$-sottospazio di $M$ generato da $X$, ovvero:
$$\bar{f}(\langle X \rangle_R) = g(\langle X \rangle_R)$$
Ma $\langle X \rangle = M$ per ipotesi, quindi $\bar{f} = g$.
\end{proof}

I moduli liberi sono la cosa più vicina agli spazi vettoriali, infatti come gli spazi vettoriali hanno una \emph{base}, che è una proprietà da cui discendono tutte le proprietà
peculiari degli spazi vettoriali.

\subsection{Prodotto tensoriale} \label{sec:prodotto-tensoriale}
\begin{defi}
\label{def: prodotto-tensoriale}
Sia $R$ un anello commutativo unitario e $V$ e $W$ due \hyperref[sec:modulo-libero]{$R$-moduli}, allora consideriamo l'$R$-modulo libero su $V \times W$ che indichiamo con $L(V, W)$.

Sia ora
\begin{multline}
R(V, W) = \langle (v_1 + v_2, w) - (v_1, w) - (v_2, w), \\
(v, w_1 + w_2) - (v, w_1) - (v, w_2), \\
(\alpha v, w) - \alpha (v, w), (v, \alpha, w) - \alpha (v, w) : v, v_1, v_2 \in V, w, w_1, w_2 \in V, \alpha \in R \rangle_R
\end{multline}

un sottomodulo di $L(V, W)$.

Allora possiamo considerare l'$R$-modulo quoziente $L(V, W)/R(V, W)$ che chiamiamo \emph{prodotto tensoriale} di $V$ e $W$ e che indichiamo col simbolo $V \otimes W$.
\end{defi}

Come gli $R$-moduli liberi, anche i prodotti tensoriali possono essere caratterizzati da una proprietà universale, per definirla però, dobbiamo prima introdurre la funzione
$\pi : V \times W \to V \otimes W$, proiezione dei generatori al quoziente, definita come segue:
$$\pi(v, w) = [(v, w)]$$

Le classi di equivalenza $[(v, w)]$ le indichiamo con il simbolo $v \otimes w$, quindi possiamo anche scrivere $\pi(v, w) = v \otimes w$. Inoltre, è facile vedere che $\pi$
è una funzione $R$-bilineare.

Dimostriamo ora che $\{ v \otimes w : v \in V, w \in W \}$ è un insieme di generatori per $V \otimes W$. Sia $p : L(V, W) \to V \otimes W$ il prolungamento $R$-lineare di $\pi$,
allora è ovvio che $p$ è la proiezione al quoziente $L(V, W)/R(V, W)$ e quindi è suriettiva. Essendo un epimorfismo $R$-lineare, essa manda un insieme di generatori di $L(V, W)$
in un insieme di generatori di $V \otimes W$, ma allora presa la base $V x W$ di $L(V, W)$, essa viene mandata da $p$ nell'insieme $\{ v \otimes w : v \in V, w \in W \}$
che risulta, quindi, essere un insieme di generatori per il prodotto tensoriale di $V$ e $W$.

\begin{lem}
Siano $V$, $W$ e $N$ tre $R$-moduli, sia $f : V \times W \to N$ una funzione $R$-bilineare, allora esiste ed è \emph{unico} un omomorfismo $R$-lineare
$\bar{f} : V \otimes W \to N$ tale che $\bar{f} \pi = f$.
\end{lem}

\begin{proof}
Per la proprietà universale dei moduli liberi (Lemma \ref{thm:prop-universale-modulo-libero}), esiste un unico omomorfismo $R$-lineare $\widetilde{f} : L(V, W) \to N$.

Ora dimostriamo che $R(V, W) \subseteq ker \widetilde{f}$: siano $v_1, v_2, v \in V$, $w_1, w_2, w \in W$ e $\alpha \in R$, siccome $f$ è $R$-bilineare, abbiamo che:

$$f(v_1 + v_2, w) = f(v_1, w) + f(v_2, w)$$
$$f(v, w_1 + w_2) = f(v, w_1) + f(v, w_2)$$
$$f(\alpha v, w) = \alpha f(v, w)$$
$$f(v, \alpha w) = \alpha f(v, w)$$

Allora per come è stata definita $\widetilde{f}$ nel Lemma \ref{thm:prop-universale-modulo-libero} e per la sua linearità, si ha che:
$$f((v_1 + v_2, w) - (v_1, w) - (v_2, w)) = f(v_1 + v_2, w) - f(v_1, w) - f(v_2, w) = 0$$

Per le medesime ragioni anche gli altri generatori di $R(V, W)$ hanno immagine nulla, ne consegue che $R(V, W) \subseteq ker \widetilde{f}$.

Ma allora possiamo definire l'omomorfismo $R$-lineare $\bar{f} : V \otimes W \to N$ come segue: $\bar{f}([x]) = [\widetilde{f}(x)]$ senza problemi di mal posizione.

Osserviamo, infine che la funzione che abbiamo definito è l'unica che rispetta tutte le condizioni del teorema; sia infatti $g : V \otimes W \to N$ un altro omomorfismo $R$-lineare
tale che $g \pi = f$, allora $g$ e $\bar{f}$ coincidono sui generatori di $V \otimes W$, ma allora, siccome sono entrambi omomorfismi $R$-lineari, devono coincidere su tutto $V \otimes W$.
\end{proof}

Se $k$ è un campo e $V$ e $W$ sono due $k$-spazi vettoriali di dimensione finita, possiamo dimostrare che anche $V \otimes W$ è un $k$-spazio vettoriale di dimensione finita, esibendone una base.

Ora mostriamo una costruzione che ci servirà per trovare una base del prodotto tensoriale di $V$ e $W$, ma che è interessante di per sé.

Dati due $k$-spazi vettoriali $V$ e $W$, possiamo considerare il $k$-spazio vettoriale $Hom_k(V, W)$ degli omomorfismi $k$-lineari tra $V$ e $W$, sappiamo che questo spazio
è ancora di dimensione finita e ha dimensione $dim(V) \cdot dim(W)$. Sappiamo anche che ogni omomorfismo $k$-lineare si può mettere in corrispondenza, una volta fissate le basi di dominio
e codominio, con una matrice $dim(V) \times dim(W)$. Ora faremo vedere come rappresentare un omomorfismo $k$-lineare nel prodotto tensoriale $V \otimes W$.

Siano $\{e_i\}_{i = 1}^n$ una base di $V$ e $\{f_i\}_{i = 1}^m$ una base di $W$. Se $f \in Hom_k(V, W)$, sappiamo che essa è univocamente determinata dai suoi valori
sulla base di $V$, quindi è univocamente determinata dai vettori (ordinati) $\{f(e_i)\}_{i = 1}^n$ di $W$, perché se $v \in V$, allora vale:
$$f(v) = f(\sum\limits_{i = 1}^n \alpha_i e_i) = \sum\limits_{i = 1}^n \alpha_i f(e_i)$$.

Allora possiamo descrivere $f$ in questo modo, dato $v \in W$, per calcolare $f(v)$ ci serve solo conoscere le componenti di $v$ nella base fissata e i vettori $\{f(e_i)\}_{i = 1}^n$.
Le componenti di $v$ le possiamo ottenere usando la base $\{\phi_i\}_{i = 1}^n$ dello spazio duale $V^{*}$ e quindi possiamo rappresentare $f$ col seguente tensore:
$$\phi_1 \otimes f(e_1) + \ldots + \phi_n \otimes f(e_n)$$
dello spazio $V^{*} \otimes W$.

Osserviamo che, viceversa, se considero il tensore $\phi \otimes w \in V^{*} \otimes W$, posso costruire una funzione lineare $f : V \to W$ come segue:
$$f(v) = \phi(v) \cdot w$$
è facile vedere che è effettivamente lineare, perché questo discende dalla linearità di $\phi$ e del prodotto esterno.

Scriviamo esplicitamente l'isomorfismo che abbiamo realizzato cominciando con $F : V^{*} \otimes W \to Hom_k(V, W)$ che definiamo sfruttando la proprietà universale dei prodotti tensoriali:

Se $\phi \in V^{*}$ e $w \in W$, allora poniamo $F(\phi \otimes w) = (\lambda v. \phi(v) \cdot w)$, abbiamo già dimostrato che il lato destro è effettivamente una funzione lineare, quindi $F$
è ben posta.

Descriviamo quindi $F^{-1} : Hom_k(V, W) \to V^{*} \otimes W$ come segue, se $f \in Hom_k(V, W)$, poniamo:
$$F^{-1}(f) = \sum\limits_{i = 1}^n \phi_i \otimes f(e_i)$$

Questa funzione è lineare, infatti:
$$F^{-1}(\alpha f + \beta g) = \sum\limits_{i = 1}^n \phi_i \otimes (\alpha f(e_i) + \beta g(e_i)) = \alpha \sum\limits_{i = 1}^n \phi_i \otimes f(e_i) + \beta \sum\limits_{i = 1}^n \phi_i \otimes g(e_i) = \alpha\cdot F^{-1}(f) + \beta \cdot F^{-1}(g)$$

Sia ora $f \in Hom_k(V, W)$, si ha che:
$$F(F^{-1}(f))(v) = F(\sum\limits_{i = 1}^n \phi_i \otimes f(e_i))(v) = \sum\limits_{i = 1}^n \phi_i(v) \cdot f(e_i) = \sum\limits_{i = 1}^n v_i \cdot f(e_i) = f(v)$$
Quindi $F(F^{-1}(f)) = f$; se invece $\phi \in V^{*}$ e $w \in W$:
$$F^{-1}(F(\phi \otimes w)) = F^{-1}(\lambda v. \phi(v) \cdot w) = \sum\limits_{i = 1}^n \phi_i \otimes w$$

Quindi effettivamente sono una l'inversa dell'altra e perciò abbiamo un isomorfismo tra $Hom_k(V, W)$ e $V^{*} \otimes W$. Questo isomorfismo, composto con l'isomorfismo tra le matrici $n \times m$
e $Hom_k(V, W)$, ci permette di trovare una base di $V^{*} \otimes W$.

sia $e_{i, j}$ la matrice tutta nulla tranne che per l'elemento di indice $(i, j)$ che è uguale a $1$. Questa matrice corrisponde alla funzione lineare
$$f(v_1 e_1 + \ldots + v_n e_n) = v_i \cdot f_j$$

Questa funzione lineare corrisponde al tensore $\phi_i \otimes f_j$. Allora la base $\{e_{i, j}\}$ viene mandata nella base $\{\phi_i \otimes f_j\}$ di $V^{*} \otimes W$.

Ricordiamo che se $V$ è un $k$-spazio vettoriale di dimensione $n$, lo possiamo identificare col duale del duale $V^{**}$, allora usando la costruzione di prima, otteniamo che
$V^{**} \otimes W$ è isomorfo a $Hom(V^{*}, W)$ e la sua base è data da $\{e_i \otimes f_j\}$, e quindi $V \otimes W$ ha dimensione $dim(V) \cdot dim(W)$.

Siano ora $f : V \to V'$  e $g : W \to W'$ due funzioni lineari, definiamo $f \otimes g : V \otimes W \to V' \otimes W'$ come segue:
$$(f \otimes g)(v \otimes w) = f(v) \otimes g(w)$$

\subsection{Algebra tensoriale} \label{sec:algebra-tensoriale}
Sia $V$ un $k$-spazio vettoriale, definiamo per induzione
$$V^{\otimes 0} = k$$
$$V^{\otimes 1} = V$$
$$V^{\otimes n} = V \otimes V^{\otimes (n-1)} = V \otimes (V \otimes (\ldots(V \otimes V)\ldots)$$

\begin{defi} \label{def:algebra-tensoriale}
Sia $V$ un $k$-spazio vettoriale, definiamo la sua \emph{algebra tensoriale} come $T(V) = \oplus_{i = 0}^{\infty} V^{\otimes i}$
\end{defi}

Diamo a $T(V)$ la struttura di una $k$-algebra, definendo un prodotto su di essa. Iniziamo definendo un prodotto tra $V^{\otimes n}$ e $V^{\otimes m}$ a valori in $V^{\otimes (n+m)}$,
cioè una funzione bilineare $f : V^{\otimes n} \times V^{\otimes m} \to V^{\otimes (n+m)}$.

Per farlo, dimostriamo che il prodotto tensoriale è associativo, cioè, se $V, W, T$ sono tre $k$-spazi vettoriali, allora esiste un isomorfismo
$f : V \otimes (W \otimes T) \to (V \otimes W) \otimes T$ definito come segue:

Sia $v \in V$, iniziamo definendo $g_v : W \otimes T \to (V \otimes W) \otimes T$ al modo seguente: $g_v(w \otimes t) = (v \otimes w) \otimes t$, $g_v$ è bilineare, quindi ben posta.
Definiamo, quindi, $f(v \otimes x) = g_v(x)$, $g_v$ è bilineare anche in $v$, infatti:
$$g_{\alpha v_1 + \beta v_2}(x) = g_{\alpha v_1 + \beta v_2}(\sum\limits_{i = 1}^k w_i \otimes t_i) = \sum\limits_{i = 1}^k ((\alpha v_1 + \beta v_2) \otimes w_i)  \otimes t_i)=$$
$$=\sum\limits_{i = 1}^k \alpha (v_1 \otimes w_i) \otimes t_i + \beta (v_2 \otimes w_i) \otimes t_i) = \alpha g_{v_1}(x) + \beta g_{v_2}(x)$$
Quindi $f$ è ben definita.
Seguendo lo stesso ragionamento si definisce $h : (V \otimes W) \otimes T \to V \otimes (W \otimes T)$ ponendo $h((v \otimes w) \otimes t) = v \otimes (w \otimes t)$ che è l'inversa.

CONTINUARE

\subsection{Algebra Esterna} \label{sec:algebra-esterna}

\begin{defi} \label{def:algebra-esterna}
Sia $V$ un $k$-spazio vettoriale, $T(V)$ la sua \hyperref[def:algebra-tensoriale]{algebra tensoriale}, consideriamo l'ideale (bilatero) di $T(V)$ definito come segue: $I(V) = ( v \otimes v : v \in V )$, allora
la $k$-algebra quoziente $\Lambda(V) = T(V) / I(V)$ è detta \emph{algebra esterna} su $V$.

Denotiamo la classe di equivalenza degli elementi del tipo $v_1 \otimes \ldots \otimes v_n$ come $v_1 \wedge \ldots \wedge v_n$.
\end{defi}

Osserviamo che, ovviamente, $v \wedge v = 0$, e da questo segue che:
$$v \wedge w + w \wedge v = v \wedge w + v \wedge v + w \wedge w + w \wedge v = (v + w) \wedge w + (v + w) \wedge v = (v + w) \wedge (v + w) = 0$$
Quindi $v \wedge w = - (v \wedge w)$.

\section{Geometria Differenziale}
Occorre prestare attenzione a varietà topologiche che si prestano ad avere una struttura più ricca che ci permette di trasportare strumenti di calcolo tipici degli spazi euclidei
su questi spazi più generali.

\begin{defi} \label{def:carta}
Sia $M$ uno spazio topologico, e sia $U$ un suo aperto e sia $V$ un aperto di $\mathbb{R}^n$, allora $\phi : U \to V$ si dice $n$-\emph{carta} se $\phi$ è un omeomorfismo.
Spesso indichiamo una carta $\phi : U \to V$ mettendo in evidenza il suo dominio e la funzione con il simbolo $(U, \phi)$, infatti il codominio si può sempre rappresentare come $\phi(U)$.
\end{defi}

\begin{defi} \label{def:carte-compatibili}
Sia $M$ uno spazio topologico e $(U, \phi)$ e $(V, \psi)$ due sue \hyperref[def:carta]{$n$-carte}, esse si dicono \emph{compatibili} se la funzione
$\phi \circ \psi^{-1} : \psi(U \cap V) \subseteq \mathbb{R}^n \to \phi(U \cap V) \subseteq \mathbb{R}^n$
è un diffeomorfismo (differenziabile con inversa differenziabile).

La funzione $\phi \circ \psi^{-1}$ si chiama \emph{cambio di carte}.
\end{defi}

\begin{defi} \label{def:atlante}
Sia $M$ uno spazio topologico, una collezione $\mathcal{A} = \{(U_i, \phi_i)\}_{i \in I}$ di \hyperref[def:carta]{$n$-carte} è un \emph{atlante} se i domini di tutte le carte
formano un ricoprimento di $M$ e in più le carte sono a due a due \hyperref[def:carte-compatibili]{compatibili}.
\end{defi}

\begin{defi} \label{def:struttura-differenziale}
Sia $M$ uno spazio topologico, una $n$-\emph{struttura differenziale} su $M$ è un \hyperref[def:atlante]{$n$-atlante} massimale su $M$.
\end{defi}

Dato uno spazio topologico $M$ e un suo $n$-atlante $\mathcal{A}$, mostriamo come è possibile determinare una unica $n$-struttura differenziale su $M$ contenente le carte di $\mathcal{A}$.

Definiamo $\mathcal{\bar{A}} = \{ (U, \psi) : (U, \psi)$ è compatibile con ogni carta di $\mathcal{A} \}$, ovviamente $\mathcal{A} \subseteq \mathcal{\bar{A}}$. Facciamo vedere che è anche un atlante.

Siano $(U, \psi), (V, \phi) \in \mathcal{\bar{A}}$ allora per ogni $p \in U \cap V$, siccome $\mathcal{A}$ è un atlante, esiste $(W, \tau) \in \mathcal{A}$ tale che $p \in W$.
Per ipotesi $(U, \psi)$ e $(V, \phi)$ sono entrambe compatibili con $(W, \tau)$, allora abbiamo che le seguenti funzioni sono diffeomorfismi:
$$\tau \circ \psi^{-1} : \psi(U \cap W) \to \tau(U \cap W)$$
$$\phi \circ \tau^{-1} : \tau(V \cap W) \to \phi(V \cap W)$$

In particolare, restringedoci all'aperto $U \cap V \cap W$, possiamo comporre i due diffeomorfismi ottenendo $\phi \circ \psi^{-1}$ che è quindi un diffeomorfismo in $p \in U \cap V$.
Per l'arbitrarietà di $p$, otteniamo, quindi che $\phi \circ \psi^{-1}$ è un diffeomorfismo in $U \cap V$. Quindi le due carte scelte sono compatibili.

Infine osserviamo che non esiste nessun altro atlante contenente $\mathcal{\bar{A}}$, infatti se $\mathcal{B}$ fosse un tale atlante, presa una carta $(U, \psi) \in \mathcal{B}$, essa
dovrebbe essere compatibile con tutte le carte di $\mathcal{A}$ dato che $\mathcal{A} \subseteq \mathcal{B}$, ma allora $(U, \psi) \in \mathcal{\bar{A}}$ e quindi $\mathcal{\bar{A}} = \mathcal{B}$.

\begin{defi} \label{def:varietà-differenziale}
Sia $M$ uno spazio topologico a \emph{base numerabile e di Hausdorff}, sia $\mathcal{A}$ una \hyperref[def:struttura-differenziale]{$n$-struttura-differenziale} su $M$, allora la coppia $(M, \mathcal{A})$
si dice $n$-\emph{varietà differenziale}.
\end{defi}

È possibile costruire strutture differenziali anche spazi topologici che non sono di Hausdorff, un classico esempio è quello della retta con due origini, definita così:
Consideriamo l'insieme $R = \mathbb{R} \sqcup \{0'\}$ con la topologia contenente come aperti tutti gli aperti di $\mathbb{R}$ e in più tutti gli insiemi $(A - \{0\}) \sqcup \{0'\}$
con $A$ aperto di $\mathbb{R}$ contenente $0$. Questa è effettivamente una topologia ed è molto facile vedere perché.

$R$ non è di Hausdorff, infatti $0$ e $0'$ non possono essere separati da due aperti disgiunti, però possiamo definire il seguente atlante:
$$\mathcal{A} = \{ (\mathbb{R}, id_{\mathbb{R}}), ((\mathbb{R} - \{0\}) \sqcup \{0'\}, f) \}$$
Dove $f(0') = 0$ e $f(x) = x\, ,\forall x \in \mathbb{R}-\{0\}$.

$f$ è un omeomorfismo, infatti se $A$ è un aperto di $\mathbb{R}$, allora $f^{-1}(A) = A$ se $0 \notin A$, altrimenti $f^{-1}(A) = (A - \{0\}) \sqcup \{0'\}$, poi, se $A$ è un aperto del dominio
della sua carta, è facile vedere che questo viene mandato in un aperto di $\mathbb{R}$.

Ovviamente le funzioni sono compatibili tra di loro, quindi questo atlante determina una struttura differenziale su $R$, la retta con due origini, che però non è, per la nostra definizione,
una varietà differenziale.

\begin{defi} \label{def:funzione-differenziabile}
Siano $M$ una $m$-varietà differenziale e $N$ una $n$-varietà differenziale, allora una funzione $F : M \to N$ è \emph{differenziabile} in $p \in M$ se esiste una carta $(U, \phi)$ di $M$
tale che $p \in U$ e una carta $(V, \psi)$ di $N$ tale che $F(U) \subseteq V$, tali che $\psi \circ F \circ \phi^{-1} : \phi(U) \to \psi(F(U))$ risulti essere $C^{\infty}$ in $\phi(p)$.

La funzione $\hat{F} = \psi \circ F \circ \phi^{-1}$ si chiama \emph{rappresentazione locale} di $F$ nelle carte $(U, \phi)$ e $(V, \psi)$.

Una funzione $F : M \to N$ è differenziabile se è differenziabile in tutti i punti di $M$ ed è un \emph{diffeomorfismo} se è differenziabile e ha un'inversa anch'essa differenziabile.
\end{defi}

Osserviamo che la definizione di differenziabilità in un punto non dipende dalle carte che si scelgono, infatti essendo i cambi carta dei diffeomorfismi, si può passare da una carta
ad un'altra senza problemi, dato che la rappresentazione locale resta differenziabile se lo era in precedenza.

\begin{defi} \label{def:varietà-analitica}
Se $M$ è uno spazio topologico e $\mathcal{A}$ una \hyperref[def:struttura-differenziale]{$2n$-struttura differenziale} su $M$ in cui i cambi carta non sono solo diffeomorfismi,
ma sono isomorfismi bianalitici, allora diremo che $\mathcal{A}$ è una $n$-struttura complessa su $M$ e se poi $M$ è di Hausdorff e anche a base numerabile, allora
$(M, \mathcal{A})$ è una $n$-varietà complessa.
\end{defi}

Analogamente si può dare una definizione di funzione analitica tra varietà analitiche.

\begin{defi} \label{def:funzione-analitica}
Siano $M$ una $m$-varietà analitica e $N$ una $n$-varietà analitica, allora una funzione $F : M \to N$ è \emph{analitica} in $p \in M$ se esiste una carta $(U, \phi)$ di $M$
tale che $p \in U$ e una carta $(V, \psi)$ di $N$ tale che $F(U) \subseteq V$, tali che $\psi \circ F \circ \phi^{-1} : \phi(U) \subseteq \mathbb{C}^m \to \psi(F(U)) \subseteq \mathbb{C}^n$
risulti essere \emph{analitica} in $\phi(p)$.

Una funzione $F : M \to N$ è analitica se è analitica in tutti i punti di $M$ ed è un \emph{isomorfismo bianalitico} se è analitica e ha un'inversa anch'essa analitica.
\end{defi}

\subsection{Esempi di varietà differenziali ed analitiche}
Mostriamo vari esempi di varietà differenziali ed analitiche.

Prima di cominciare, nel seguito ci troveremo quasi esclusivamente a considerare sottoinsiemi di $\mathbb{R}^n$, facciamo vedere che questi insiemi, se muniti della topologia di sottospazio,
sono ancora spazi di Hausdorff a base numerabile:

Sia $X \subseteq \mathbb{R}^n$ e siano $p, q \in X$, allora esistono due aperti di $\mathbb{R}^n$ disgiunti, $U$ e $V$, tali che $p \in U$ e $q \in V$, ma allora $U \cap X$ e $V \cap X$
sono aperti di $X$ e sono ancora disgiunti e si ha ancora che $p \in U \cap X$ e $q \in V \cap X$, quindi $X$ è uno spazio di Hausdorff.

Sappiamo che $\mathbb{R}^n$ è uno spazio topologico a base numerabile, sia allora $\mathcal{B} = \{ B_i \}_{i \in \mathbb{N}}$ una sua base, si capisce subito che essendo il generico aperto di $X$
un insieme del tipo $A \cap X$, con $A$ aperto di $\mathbb{R}^n$, allora $\mathcal{B} = \{ B_i \cap X \}_{i in \mathbb{N}}$  è una base della topologia indotta su $X$, che quindi è ancora
a base numerabile.

\subsubsection{Circonferenza}

Procediamo dunque a considerare la circonferenza $S^1 = \{ z \in \mathbb{C} : |z|^2 = 1 \}$ ed esibiamo una struttura di varietà differenziale su di essa.

Consideriamo le due carte $(U, \phi)$ e $(V, \psi)$ definite come segue,

poniamo $U = S^1 - \{1\}$ e $V = S^1 - \{-1\}$ e poi definiamo $\phi : U \to ]0, 2\pi[$ e $\psi : V \to ]-\pi, \pi[$ così:

$\phi(z) = \theta$, dove $\theta$ è la determinazione dell'argomento di $z$ in $[0, 2\pi]$.

$\psi(z) = \theta$, dove $\theta$ è la determinazione dell'argomento di $z$ in $[-\pi, \pi]$.

Sono entrambe funzioni invertibili, le inverse sono date da:

$\phi^{-1}(\theta) = e^{i\theta}$ e $\psi^{-1}(\theta) = e^{i\theta}$. Osserviamo che la funzione $(z \in \mathbb{C} \mapsto e^{iz})$ è una funzione olomorfa, quindi è una funzione aperta, restringendo
il suo dominio agli insiemi di definizione di $\phi^{-1}$ e $\psi^{-1}$, essa è iniettiva, quindi resta ancora una funzione aperta come si può facilmente verificare. Di conseguenza,
$\phi^{-1}$ e $\psi^{-1}$ sono funzioni continue aperte e invertibili, ossia degli omeomorfismi.

I domini delle due carte ricoprono tutta la circonferenza, quindi non ci resta che dimostrare che le due carte sono tra loro compatibili:
Consideriamo, $\psi \circ \phi^{-1} : \phi(S^1 - \{1, -1\}) \to \psi(S^1 - \{1, -1\})$,

osserviamo che $\phi(S^1 - \{1, -1\}) = ]0, \pi[ \cup ]\pi, 2\pi[$ e che $\psi(S^1 - \{1, -1\}) = ]-\pi, 0[ \cup ]0, \pi[$, quindi:

$$
(\psi \circ \phi^{-1})(\theta) = \begin{cases}
    \theta &\text{se $\theta \in ]0, \pi[$}\\
    \theta - 2\pi &\text{se $\theta \in ]\pi, 2\pi[$}
\end{cases}
$$

e similmente,

$$
(\phi \circ \psi^{-1})(\theta) = \begin{cases}
    \theta &\text{se $\theta \in ]0, \pi[$}\\
    \theta + 2\pi &\text{se $\theta \in ]-\pi, 0[$}
\end{cases}
$$

È chiaro che il cambio di carte è un diffeomorfismo, abbiamo perciò definito una struttura di $1$-varietà differenziale sulla circonferenza, che per quanto visto in precedenza
è anche uno spazio topologico di Hausdorff a base numerabile, quindi è una varietà differenziale.

\subsubsection{Sfera di Riemann}

Costruiamo ora un primo esempio di varietà non solo differenziale, ma anche analitica. Sia $\hat{\mathbb{C}} = \mathbb{C} \cup \{\infty\}$ il piano complesso esteso.

La topologia di questo spazio è ottenuta come compattificazione all'infinito dell'insieme dei numeri complessi. Definiamo le seguenti due carte:

Sia $U = \mathbb{C}$ e $\phi(z) = z$, sia $V = \hat{\mathbb{C}} - \{0\}$ e $\psi(z) = \frac{1}{z}$, dove si intende che $\frac{1}{\infty} = 0$.

È chiaro che entrambe le funzioni sono omeomorfismi, e che i cambiamenti di carta sono isomorfismi bianalitici, infatti,

$\phi \circ \psi^{-1} (z) = \frac{1}{z}$ è bianalitica in $\mathbb{C} - \{0\}$.

\subsubsection{Grafici di funzioni}
Sia $U$ un aperto di $\mathbb{R}^n$ e sia $f : U \to \mathbb{R}$ una funzione continua. Consideriamo il suo grafico $G_f = \{ (x, f(x)) \in U \times \mathbb{R} \}$.
Allora possiamo dotare $G_f$ di una struttura di varietà differenziabile. In questo caso ci basta una sola carta, definita come la proiezione sull'piano orizzontale:

Sia $\pi : G_f \to \mathbb{R}^n$ definita come $\pi(x, y) = x$, allora è ovviamente una funzione continua, ma è anche invertibile, con inversa $\pi^{-1}(x) = (x, f(x))$,
anch'essa continua per l'ipotesi che ci garantisce la continuità di $f$. Quindi abbiamo una carta, una carta globale, che quindi ci permette di definire un $n$-atlante.

\subsubsection{Insiemi di livello}
Siamo in grado di costruire una struttura differenziale anche su insiemi di livello di funzioni differenziabili. Questi insiemi, per il teorema della funzioni implicita, possono essere visti
localmente come grafici di funzioni.

Sia $F : \mathbb{R}^n \to \mathbb{R}$ una funzione $C^{\infty}$, e sia $X = \{ p \in \mathbb{R}^n : F(p) = 0 \}$, se per ogni $p \in X$, esiste $i \in \{1,\ldots,n\}$ tale che $\frac{dF}{dx_i}(p) \neq 0$,
allora per il teorema della funzione implicita, $X$ è localmente il grafico di una funzione, ovvero:

Per ogni $p \in X$, esiste un aperto $A \subseteq \mathbb{R}^n$ tale che $X \cap A$ è il grafico di una funzione,  quindi esiste un aperto $U \subseteq \mathbb{R}^{n-1}$ e una funzione
$h : U \to \mathbb{R}$ liscia e un indice $i$ (corrispondente all'indice tale che $\frac{dF}{dx_i}(p) \neq 0$) tali che:

$$X \cap A = \{ (x_1, \ldots, x_{i-1}, h(\underbar{x}), x_{i+1}, \ldots, x_n) : \underbar{x} \in U \}$$

Allora la proiezione $\pi_i : X \cap A \to U$ è un omeomorfismo, visto che ha per inversa $\pi_i^{-1}(x_1, \ldots, \hat{x_i}, \ldots, x_n) = (x_1, \ldots, x_{i-1}, h(\underbar{x}), x_{i+1}, \ldots, x_n)$ anch'essa continua.

Quindi $(X \cap A, \pi_i)$ è una carta e quindi abbiamo definito una collezione di carte i cui domini ricoprono $X$, mostriamo che le carte sono anche compatibili tra di loro:

Siano $(X \cap A, \pi_i)$ e $(X \cap B, \pi_j)$ due carte tali che $X \cap A \cap B \neq \emptyset$, se $i = j$, allora il cambio carta è l'identità, quindi è liscia,
se invece $i \neq j$, per esempio $i < j$, allora si ha:

$$\pi_i \circ \pi_j^{-1} (x_1, \ldots, x_{n-1}) = (x_1, \ldots, x_{i-1}, x_{i+1}, \ldots, x_{j-1}, h(\underbar{x}), x_{j+1}, \ldots, x_n)$$

Che è una funzione liscia. Quindi quello che abbiamo definito è un atlante e $X$ è una $(n-1)$-varietà differenziale.

Analogamente se $F : \mathbb{R}^n \to \mathbb{R}^m$ è una funzione differenziabile, con $m \le n$ e $X = V(F) = \{ p \in \mathbb{R}^n : F(p) = 0 \}$ è tale che, per ogni $p \in X$,
il rango di $F$ è massimo, allora sempre sfruttando il teorema della funzione implicita, si può vedere $X$ localmente come grafico di una funzione e come fatto prima, si può definire
una struttura di $(n-m)$-varietà differenziale.

Consideriamo un ultimo caso.

Se $F : \mathbb{C}^2 \to \mathbb{C}$ è una funzione analitica, Possiamo scrivere $F$ nel seguente modo, $F(x, y, a, b) = u(x, y, a, b) + iv(x, y, a, b)$, e siccome $F$ è analitica,
per le formula di Cauchy-Riemann, si ha che:
$$\frac{\partial(u, v)}{\partial(x, y)} = \left|\frac{dF}{dz}\right|^2$$
e
$$\frac{\partial(u, v)}{\partial(a, b)} = \left|\frac{dF}{dw}\right|^2$$

Ora, sia $X$ il suo insieme degli zeri, se per ogni $p \in X$ si ha $\frac{dF}{dz}(p) \neq 0$ oppure $\frac{dF}{dw}(p) \neq 0$, uno dei due determinanti sopra è non nullo,
ma quindi la matrice Jacobiana di $F$ ha un minore di ordine 2 non nullo, quindi ha rango almeno 2, ma 2 è anche il rango massimo possibile, essendo lo Jacobiano una matrice $2 \times 4$.

Quindi, possiamo applicare il teorema della funzione implicita per costruire una struttura differenziale su $X$. Ma dato che $F$ non è solo liscia, ma è anche analitica, è possibile
che l'atlante definito non dia solo una struttura differenziale, ma anche una struttura analitica?

\subsection{Sottovarietà}

\end{document}

